% Packages
\documentclass[12pt]{article}
\usepackage[utf8]{inputenc}
\usepackage[english]{babel}
\usepackage{amsmath} 
\usepackage{pdfpages}
\usepackage{csvsimple}
\usepackage{amsfonts}
\usepackage{mathrsfs}
\usepackage{bbm}
\usepackage{pdfpages}
\usepackage{graphicx}
\usepackage{booktabs}
\usepackage{geometry}
\usepackage{subcaption}
\usepackage{float}

% Commands
\newcommand{\E}{\mathbb{E}}
\newcommand{\R}{\mathbb{R}}
\newcommand{\Cov}{\mathrm{Cov}}
\newcommand{\Var}{\mathrm{Var}}
\newcommand{\indep}{\perp \!\!\! \perp}
\newcommand{\1}{\mathbbm{1}}
\newcommand{\pr}{\mathbb{P}}
\newcommand{\X}{\mathcal{X}}
\newcommand{\Y}{\mathcal{Y}}
\newcommand{\e}{\varepsilon}

% formatting
\geometry{legalpaper, portrait, margin=1in}


% Title
\title{IO II- Pset 3}
\author{Tom Hierons \& Alexander Petrov }
\date\today
\setlength{\parindent}{0pt}

\begin{document}
\maketitle 
\section*{1 Data}
\subsection*{1.}
Yes.
\subsection*{2.}
We replicate table 1 below using the method outlined in the notes. Please see the code for further details. The sample data does not seem to have the required variables to provide the final column. The findings are broadly in line with those in the paper given that this is a 40\% subsample.
\begin{table}[H]
    \centering
    \caption{Replicating Table 1}
    \begin{tabular}{rrrrrrr}
\toprule
 Target auction mean &  Target auction SD &  Knockout auction mean &  Knockout auction SD &  \% lots won &  \% total est. value won &  number of lots \\
         1530.615385 &        1571.676928 &            1111.538462 &          1560.452890 &    0.827586 &                0.565333 &              29 \\
\midrule
         1174.455357 &        1606.273679 &            1330.357143 &          2756.704743 &    0.405405 &                0.657061 &              37 \\
         1388.400493 &        1932.833336 &            1383.759046 &          2269.849870 &    0.491525 &                0.442273 &             385 \\
         1424.003497 &        1583.670665 &            1326.643357 &          1757.031266 &    0.660131 &                0.594296 &             153 \\
         3140.533118 &        2918.238243 &            4219.531502 &          4825.892059 &    0.228346 &                0.136652 &             127 \\
         2036.286982 &        3057.300653 &            1941.035503 &          3706.332154 &    0.620000 &                0.508270 &             100 \\
\bottomrule
\end{tabular}

    \label{tab:my_label}
\end{table}

Table 2 is also replicated. For the knockout auction we take the mean of the median bids grouped by number of bidders. Further details are in the attached code. Again the findings are broadly in line with the paper's. 
\begin{table}[H]
    \centering
    \caption{Replicating Table 2}
    \begin{tabular}{lrrrrrr}
\toprule
{} &  Target mean &  Target SD &  Bid mean &  Bid SD &  \% lots won &  \# lots \\
\# Bidders &              &            &           &         &             &         \\
\midrule
1         &       566.78 &     861.98 &    496.91 &  898.71 &        0.28 &     267 \\
2         &       990.82 &    1803.20 &    946.21 & 2595.65 &        0.59 &     149 \\
3         &      1221.42 &    1512.19 &   1041.38 & 1516.46 &        0.68 &     127 \\
4         &      2029.59 &    2824.86 &   2129.30 & 4035.05 &        0.77 &     110 \\
5         &      1754.94 &    2331.02 &   1715.57 & 2574.73 &        0.84 &      79 \\
6         &      2327.45 &    3144.88 &   2584.68 & 3656.21 &        0.89 &      55 \\
7         &      2722.16 &    2052.97 &   3280.00 & 2509.41 &        0.96 &      51 \\
8         &      4290.48 &    2555.56 &   5395.24 & 3669.88 &        0.95 &      21 \\
\bottomrule
\end{tabular}

    \label{tab:my_label}
\end{table}
\subsection*{3.}
Done. 

\section*{2 Introductory Questions}
\subsection*{1.}
The presence of side-payments induces an incentive to lie upwards about your valuation: you receive a share of the surplus provided that your bid is above the price at which the item is sold even if you are not the highest bidder. This is established formally in result 1 of the paper.

\subsection*{2.}
\subsubsection*{(a)}
For the case of two players this associates a unique valuation with any given bid. It maps bids and bid distributions to valuations. 

\subsubsection*{(b)}
This shows that the same valuation may be mapped to by multiple bids. This is due to the fact that $\frac{\partial v(b)}{\partial b}$ depends on the distribution functions of other bidders both within and outside the ring in a way that makes it impossible to sign. In terms of identification this means that estimates identify bids with valuation corresponding to local minima rather than optimal play. 

\subsubsection*{(c)}
The authors assume the true data generating process results in a monotonic valuation function so that Lemma 2 fails in the empirically relevant case.   

\subsection*{3.}
Limiting the analysis to two bidders ensures identification of the model. In particular, lemma 3 only applies to two bidder knockout auctions. 

\subsection*{4.}
\textbf{Sellers.} They can be worse or better off. The collusion by ring members tends to force the price down relative to the bids in the knockout stage leading to a loss in seller revenue. However, the knockout bids themselves are subject to incentives for overbidding which tends to raise seller revenue.  \\

\textbf{Non-ring bidders.} These agents are unambiguously worse off. Higher bids from ring bidders leads them to lose some auctions that they would otherwise have won.\\

\textbf{Ring members.} The effect here is also ambiguous. The incentives to overbid may lead ring-members to be hurt by the knockout stage auction. However, collusion among the ring members also allows them to appropriate more of the surplus either directly or via sidepayments. The latter effect appears to dominate empirically. \\

\section*{5.}
\begin{table}[H]
	\centering
	\caption{Replicating Table 5}
	\begin{tabular}{lrrrrrr}
\toprule
{} &  \% KO won &  \# KOs &  \% KO won &  \% receive side &  \% pay side &  \# KOs \\
Bidder \# &           &        &           &                 &             &        \\
\midrule
1        &      0.40 &    652 &      0.24 &            0.23 &        0.08 &    517 \\
2        &      0.10 &    362 &      0.08 &            0.24 &        0.02 &    357 \\
3        &      0.38 &    394 &      0.27 &            0.29 &        0.10 &    339 \\
4        &      0.27 &    323 &      0.26 &            0.29 &        0.14 &    318 \\
5        &      0.38 &    269 &      0.23 &            0.24 &        0.06 &    219 \\
6        &      0.46 &    153 &      0.46 &            0.14 &        0.20 &    152 \\
7        &      0.41 &    232 &      0.40 &            0.26 &        0.24 &    227 \\
8        &      0.17 &     30 &      0.04 &            0.35 &        0.00 &     26 \\
9        &      0.48 &     77 &      0.45 &            0.29 &        0.16 &     73 \\
10       &      0.09 &     99 &      0.07 &            0.42 &        0.04 &     97 \\
11       &      0.20 &     45 &      0.18 &            0.48 &        0.02 &     44 \\
\bottomrule
\end{tabular}

	\label{tab:my_label}
\end{table}

\end{document}